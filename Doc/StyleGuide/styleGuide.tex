\documentclass[11pt]{article}
\setlength{\textwidth}{6.5 in}
\setlength{\oddsidemargin}{-0. in}
\setlength{\evensidemargin}{\oddsidemargin}

\title{Style guide for PoPI, GIDI and MCGIDI: C++ APIs for reading and sampling GNDS data\footnote{
This work was performed under the auspices of the U.S. Department of Energy by
Lawrence Livermore National Laboratory under contract \#W-7405-ENG-48.}}
\author{{Bret R. Beck}\\Lawrence Livermore National Laboratory\\UCRL-}

\begin{document}

\maketitle

\pagebreak
\tableofcontents
\listoftables
\pagebreak

\section{Introduction}
This document describes the style guide used for the PoPI, GIDI and MCGIDI C++ APIs.

Indentation is four spaces. Tabs are not allowed anywhere. If a line is being continued, two indentation are required on the
indented lines.

\section{Naming}
    As a general rule, names should be descriptive. Abbreviation are discouraged (e.g., use \texttt{crossSection} and not \texttt{xSec}).
One can use acronyms (e.g., \texttt{NASA}, \texttt{ASCII}, \texttt{GIDI}); however, all letters in the acronym must be upper cased and
all acronyms must be separeted by an underscore character (i.e., "\_") from other parts of the name.
For example, use \texttt{GIDI\_neutronName} instead of \texttt{GIDIneutronName}, \texttt{GIDINeutronName} or \texttt{gidiNeutronName}.
Most names are either \texttt{camelCase} (i.e., all first letters of the words are capitalized except for the first word),
\texttt{PascalCase} (i.e., all first letters of the words are capitalized), or \texttt{camelCase} preceded by either \texttt{m\_} (for member names)
or \texttt{a\_} (for argument names).

Instance names should be the camel case representation of the object's name. For example, an instance of the \texttt{Reaction} class will have the
name \texttt{reaction}.  If more than one instance of an object is needed within a section, names are constructed by adding sequence numbers to
the prior rule starting with the number 1. For example, if three instances of a \texttt{Reaction} class are needed within a method, the names will be \texttt{reaction1},
\texttt{reaction2} and \texttt{reaction3}.

For an integer-type indexing variable in a \texttt{for} loop, the name should be as the ojbect being looped over with the extension \texttt{Index}. For example,
when looping over a list of \texttt{Reaction} class objects, the integer-type indexing variable should be \texttt{reactionIndex}.
If the loop is short, one can use the integer-type indexing variable \texttt{i\#} where the "\#" represents an integer string. For example, \texttt{i1},
\texttt{i2} or \texttt{i3}.

\begin{description}
    \item[class:] Class names are pascal case. Examples are \texttt{Protare}, \texttt{Reaction} and \texttt{HeatedCrossSectionContinuousEnergy}.
    \item[member:] Member names begin with a \texttt{m\_}, followed by a camel case. Examples are \texttt{m\_domainMin}, 
                    \texttt{m\_crossSection} and \texttt{m\_reactions}.
    \item[method:] Method names are camel case. Examples are \texttt{crossSection}, \texttt{sampleReaction} and \texttt{serialize}.
    \item[function:] Same rule as for method names.
    \item[arument:] Argument names begin with a \texttt{a\_}, followed by a camel case. Examples are \texttt{a\_domainMin},
                    \texttt{a\_crossSection} and \texttt{a\_reactions}.
\end{description}

\section{if-else statements}
All single statment \texttt{if} statements must of the form
\begin{verbatim}
    if( test ) statement
\end{verbatim}

All \texttt{if-else} statements must of the form
\begin{verbatim}
    if( test ) {
        statement;
        ...
        statement; }
    else if( test ) {
        statement;
        ...
        statement; }
    else if( test ) {
        statement;
        ...
        statement; }
    else {
        statement;
        ...
        statement;
    }

\end{verbatim}

\section{binary and logical operators}
All binary and logical operators must have at least one space before the operator and at least one space after the operator. 
This rule also applies to the equal sign (i.e., "=").

\begin{verbatim}
    sum = part1 + part2 + part3     // One indentation.
            part4 + part5 +         // Three indentations.
            part6                   // Three indentations.
\end{verbatim}
If alignment is desired, additional spaces are allowed around a binary operator. For example, the prior example can be written as
\begin{verbatim}
    sum =   part1 + part2 + part3   // One indentation.
            part4 + part5 +         // Three indentations.
            part6                   // Three indentations.
\end{verbatim}

\end{document}
