\subsection{Functions}
This section decribes all the functions in the file ``ptwXY\_functions.c''.

\subsubsection{ptwXY\_pow}
\setargumentNameLengths{ptwXY}
This function applies the math operation $y_i = y_i^{p}$ to the y-values of \highlight{ptwXY}.
\CallingC{fnu\_status ptwXY\_pow(}{statusMessageReporting *smr,
    \addArgument{ptwXYPoints *ptwXY,}
    \addArgument{double p );}}
    \argumentBox{smr}{The \highlight{statusMessageReporting} instance to record errors.}
    \argumentBox{ptwXY}{A pointer to the \highlight{ptwXYPoints} object.}
    \argumentBox{p}{The exponent.}
    \vskip 0.05 in \noindent
This function infills to maintain the initial accuracy.

\subsubsection{ptwXY\_exp}
\setargumentNameLengths{ptwXY}
This function applies the math operation $y_i = \exp( a \, y_i )$ to the y-values of \highlight{ptwXY}.
\CallingC{fnu\_status ptwXY\_exp(}{statusMessageReporting *smr,
    \addArgument{ptwXYPoints *ptwXY,}
    \addArgument{double a );}}
    \argumentBox{smr}{The \highlight{statusMessageReporting} instance to record errors.}
    \argumentBox{ptwXY}{A pointer to the \highlight{ptwXYPoints} object.}
    \argumentBox{a}{The exponent coefficient.}
    \vskip 0.05 in \noindent
This function infills to maintain the initial accuracy.

\subsubsection{ptwXY\_convolution}
This function returns the convolution of \highlight{ptwXY1} and \highlight{ptwXY2}.
\setargumentNameLengths{ptwXY1}
\CallingC{ptwXYPoints *ptwXY\_convolution(}{statusMessageReporting *smr,
    \addArgument{ptwXYPoints *ptwXY1,}
    \addArgument{ptwXYPoints *ptwXY2,}
    \addArgument{int mode );}}
    \argumentBox{smr}{The \highlight{statusMessageReporting} instance to record errors.}
    \argumentBox{ptwXY1}{A pointer to a \highlight{ptwXYPoints} object.}
    \argumentBox{ptwXY2}{A pointer to a \highlight{ptwXYPoints} object.}
    \argumentBox{mode}{Flag to determine the initial x-values for calculating the convolutions.}
    \vskip 0.05 in \noindent
User should set \highlight{mode} to 0. 

\subsubsection{ptwXY\_inverse}
This function returns a new instance of \highlight{ptwXYPoints} that is the inverse of \highlight{ptwXY1}.
That is, the returned points are $(y_{\rm i},x_{\rm i})$ where $(x_{\rm i},y_{\rm i})$ are the points of \highlight{ptwXY1}.
All y-values of \highlight{ptwXY1} must be descending or increasing. If the y-values of \highlight{ptwXY1} are descending,
the returned points are reversed to insure that in the returned instance $X_{\rm i} < X_{\rm i+1}$ where $X_{\rm i}$ is a 
x-value of new returned instance.
\setargumentNameLengths{ptwXY1}
\CallingC{ptwXYPoints *ptwXY\_inverse(}{statusMessageReporting *smr,
    \addArgument{ptwXYPoints *ptwXY1 );}}
    \argumentBox{smr}{The \highlight{statusMessageReporting} instance to record errors.}
    \argumentBox{ptwXY1}{A pointer to a \highlight{ptwXYPoints} object.}
    \argumentBox{ptwXY2}{A pointer to a \highlight{ptwXYPoints} object.}
    \argumentBox{mode}{Flag to determine the initial x-values for calculating the convolutions.}
    \vskip 0.05 in \noindent
If an error occurs, NULL is returned.
